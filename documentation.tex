\documentclass[parskip=half*]{scrartcl}
\usepackage[spanish, activeacute]{babel}
\usepackage{graphicx}
\usepackage{makeidx}
\usepackage{moreverb}

\title{Dise\~no de Software para Dispositivos M\'oviles}
\author{Jos\'e Ladislao Lainez Ortega y Jos\'e Molina Colmenero}
\makeindex

\begin{document}

% FRONT PAGE

\maketitle
\tableofcontents

\newpage

% MAIN DOCUMENT

\section{Introducci\'on}
El siguiente documento muetra la informaci\'on relativa al dise\~no e implementaci\'on de una aplicaci\'on de gesti\'on de tareas para iOS, como parte de la asignatura Dise\~no de Software para Dispositivos M\'oviles del Grado de Ingenier\'ia inform\'atica de la Universidad de Ja\'en.

\section{Concepto}
Para dar algo de originalidad al proyecto vamos a usar una aproximaci\'on distinta al sistema de tareas tradicional, de forma que podamos aplicar esta nueva filosof\'ia en el dise\~no de la aplicaci\'on desde el principio.

\subsection{Getting Things Done}
O com\'unmente conocida como GTD es una metodolog\'ia de organizaci\'on de tareas enfocada a la productividad. Sus pilares clave son la sencillez y la categorizaci\'on, de forma que el usuario no tiene que aprender complejos patrones ni adaptar dr\'asticamente su forma de organizarse.

El funcionamiento base de GTD persigue que el usuario se olvide de las tareas hasta el momento en que tiene que hacerlas, de forma que pueda concentrarse en aquellas tareas que est\'an m\'as cerca en el tiempo. Para ello se hace uso de varias categor\'ias que clasificar\'an las tareas seg\'un unos criterios de cercan\'ia en el tiempo y que ser\'an revisadas una vez al d\'ia.

Para conseguir que el usuario no tengan en mente las tareas que tiene que hacer y se pase el d\'ia pensando ``Que no se me olvide esto, cuando llegue a casa tengo que hacerlo" GTD propone que la creaci\'on de tareas sea extremadamente sencilla. La idea consiste en crear las tareas justo en el momento en que las piensas, pero solo especificando un t\'itulo para esta; no ser\'a necesario especificar categor\'ia, fecha ni ning\'un otro par\'ametro, el objetivo en ese momento es que el usuario cree la tarea y se olvide de ella para que no siga dando vueltas en su cabeza, ya llegar\'a el momento de atenderla. Estas tareas reci\'en creadas van a la categor\'ia llamada `Inbox'.

A lo largo del d\'ia el usuario habr\'a creado varias tareas que ahora est\'an en la categor\'ia `Inbox'. \¿Qu\'e hace con estas tareas? Pues buscar un momento al d\'ia en el que revisar las tareas que se encuentran en `Inbox' y asignarle una nueva categor\'ia aparte de proporcionar la informaci\'on adicional que sea necesaria.

En este punto hay varias versiones de GTD que usan categor\'ias distintas pero nosotros nos quedaremos en la m\'as b\'asica para mantener la aplicaci\'on lo m\'as sencilla de usar, por lo que tendremos las siguientes categor\'ias para asignar a nuestras tareas del `Inbox' seg\'un corresponda a cada una:

\begin{description}
	\item[Next] \hfill \\ Aquellas tareas que se pueden realizar pr\'oximamente. ``Comprar el pan''.
	\item[Waiting] \hfill \\ Tareas que podr\'ian pertenecer a `Next' pero dependemos de alguien o algo para poder realizarlas. ``Ir de picnic un d\'ia soleado''.
	\item[Project] \hfill \\ Relacionadas con alg\'un proyecto en el que el usuario est\'a involucrado. ``Corregir el fallo en la interfaz de la aplicaci\'on para DSDM''.
	\item[Someday] \hfill \\ Cosas que se quieren realizar alg\'un d\'ia, planes a largo plazo. ``Comprar un MacBook Pro Retina Display''.
\end{description}

De esta forma hemos dividido en dos etapas el flujo de trabajo; durante la primera se crean las tareas, que quedar\'an almacenadas en un buz\'on (`Inbox') para que no tengamos que pensar en ellas durante todo el d\'ia y la segunda en la que revisamos las tareas para actualizar su informaci\'on ya sea cambi\'andolas de categor\'ia, marc\'andolas como completadas o elimin\'andolas si ya no son necesarias.

\section{Dise\~no}

Siguiendo la filosof\'ia anterior dise\~naremos nuestra aplicaci\'on para que se amolde al esquema de funcionamiento de GTD, procurando que sea lo m\'as sencillo posible.

\subsection{Tareas}

Las tareas estar\'an compuestas por los siguientes atributos:
\begin{itemize}
	\item Nombre.
	\item Descripcion o notas.
	\item Prioridad.
	\item Fecha de creaci\'on.
	\item Categor\'ia.
\end{itemize}

Al crear la tarea esta recibir\'a la categor\'ia `Inbox' autom\'aticamente, pudiendo ser esta modificada m\'as adelante durante la revisi\'on de las tareas. Los atributos `prioridad' y `fecha de creaci\'on' se han a\~nadido por conveniencia a pesar de no ser necesarios en la metodolog\'ia GTD para aportar m\'as opciones de ordenaci\'on a la hora de visualizar el listado de tareas de una categor\'ia.

\subsection{Categor\'ias}

Ya hemos comentado antes una serie de categor\'ias relacionadas con GTD, pero hemos de a\~nadir otras dos m\'as para contemplar dos posibles estados de las tareas; completadas y borradas.

\begin{itemize}
	\item Inbox.
	\item Next.
	\item Waiting.
	\item Project.
	\item Someday.
	\item Completed.
	\item Trash. 
\end{itemize}

Las categor\'ias `Completed' y `Trash' son asignadas a las tareas completadas y eliminadas respectivamente, de forma que podamos listar estas tareas en nuestra aplicaci\'on. Aunque estas categor\'ias no entren dentro de GTD eran necesar\'ias por los requisitos impuestos para el proyecto.

\subsection{Interfaz}

La funcionalidad que debe otorgar la aplicaci\'on debe ser directa y sencilla para que la creaci\'on de tareas sea lo m\'as \'agil posible y debemos poder ver las tareas por categor\'ias a la hora de revisarlas y editarlas durante la revisi\'on diaria de tareas.

Por tanto la aplicaci\'on debe contar con las siguientes vistas:

\begin{description}
	\item[Home] \hfill \\ Acceso directo al `Inbox' y a la creaci\'on de tareas, as\'i como a las categor\'ias.
	\item[Creaci\'on de tarea] \hfill \\ Permite crear una tarea nueva.
	\item[Categor\'ia] \hfill \\ Listado de las tareas pertenencientes a una categor\'ia. Debe permitir ordenar la lista seg\'un varios criterios.
	\item[Tarea] \hfill \\ Visualzaci\'on de la informaci\'on de una tarea concreta. Se debe poder editar la informaci\'on de la tarea para realizar, por ejemplo, el cambio de categor\'ia. Tambi\'en se debe poder marcar la tarea como completada o borrarla si ya no es necesaria.
	\item[Edici\'on]	\hfill \\ Edici\'on de la informaci\'on de la tarea. Se debe poder escoger la categor\'ia a la que asignarla.
\end{description}

\subsection{Notificaciones}

Aunque en un principio pueda parecer que insertar notificaciones para las tareas individuales que se encuentran en `Inbox' es una buena idea, rompe con un concepto fundamental de GTD: que el usuario deje de pensar en la tarea tras crearla y hasta que revise el `Inbox'. Una buena idea de notificaci\'on ser\'ia una que avisara al usuario que tiene tareas sin clasificar en la categor\'ia `Inbox' cada cierto tiempo (unas 12-16 horas estar\'ia bien), de forma que este no olvide revisar sus tareas al menos una vez al d\'ia, pero sin saturarlo.

\section{Implementaci\'on}

Para la implementación de la aplicación se ha seguido el patrón Modelo Vista Controlador (MVC) de forma que:

	\item[Modelo] \hfill \\ Está constituido por todos los objetos que representan la información. Esto incluye tanto las clases utilizadas por la aplicación durante su ejecución como las creadas únicamente para la realización de la persistencia de datos. Asimismo, toda la funcionalidad de gestión de los datos se lleva a cabo en una clase a parte para poder manejar la información de forma más sencilla, tal y como recomienda Apple. Esta clase engloba toda la funcionalidad requerida para la inserción, borrado, modificación y consulta de datos.
	\item[Vista] \hfill \\ Los objetos de vista son aquellos que se muestran al usuario como presentación de los datos, es decir, las distintas interfaces con las que el usuario puede interactuar. Los objetos de vista no poseen en sí capacidad de procesar información, sino que son los objetos controladores los encargados de de esta tarea.
	\item[Controlador] \hfill \\ Los objetos de controlador son aquellos que procesan la información que el usuario aporta, la procesan y, mediante consultas al modelo, muestran al usuario una respuesta. Los controladores se encargan de manejar la lógica de la aplicación, y serían los objetos que controlan cada una de las vistas.

\subsection{Tareas}

Las tareas pertenecen al grupo de clases del Modelo, y almacenan todos los datos concernientes a esa entidad. Los objetos tarea se utilizan como fuentes de información para poder mostrar los datos de una vista, de forma que cada vista en la aplicación necesita de al menos un objeto tarea del que poder extraer la información.

\subsection{Categor\'ias}

Las categorías también forman parte del Modelo, dado que se usan para clasificar y organizar las tareas según ese criterio.

\subsection{Interfaz}

Todos los objetos que pertenecen a la interfaz de usuario a su vez serán clases de Vista. En la interfaz se muestra de forma clara al usuario la información mediante objetos gráficos con los que pueda interactuar mandando información a los controladores de los eventos producidos. Según el tipo de dato a mostrar o editar se utilizan varios tipos de objetos: UITextView para editar campos de texto y UILabel para mostrarlos; UITableView para mostrar la información de forma ordenada por filas y poder seleccionar las celdas, Sliders para la introducción de campos numéricos, UIButtons para la selección de las acciones, etc. La principal forma de organización de la información ha sido mediante tablas, de forma que la información se presenta en cada una de las celdas de las que dispone la tabla. A su vez estas tablas podían ser de contenido estático o dinámico, dependiendo del caso.

\subsection{Controladores}

Según el tipo de vista se han implementado dos tipos de clases de controlador: clases tipo UITableViewController y clases UIViewController. Cada una de ellas se encarga de manejar, atendiendo al tipo de eventos que se pueden producir, las acciones realizadas por el usuario. A su vez, para cambiar de una vista a otra (y por tanto de un controlador a otro) se ha utilizado la funcionalidad que proveen los storyboards en iOS 5+, más concretamente mediante los llamados `segues'.

\subsection{Notificaciones}

A pesar de la idea propuesta en la fase de dise\~no, para poder mostrar la implementaci\'on de las notificaciones y que el profesor no tenga que esperar varias horas como se propon\'ia se ha hecho que aparezca una notificaci\'on 20 segundos despu\'es de iniciar la aplicaci\'on.

% Code snippet example
\begin{verbatimtab}
-(void)scheduleNotification
{
    UILocalNotification *notification = [[UILocalNotification alloc] init];
    
    // notification is going to fire after 20 seconds
    notification.fireDate = [[NSDate alloc] initWithTimeIntervalSinceNow:20];
    
    // message to show
    notification.alertBody = @"Remeber to take a look at your uncategorized task!";
    
    // default sound
    notification.soundName = UILocalNotificationDefaultSoundName;
    
    // button tittle
    notification.alertAction = @"Go";
    notification.hasAction = YES;
    
    [[UIApplication sharedApplication] scheduleLocalNotification:notification];
}
\end{verbatimtab}

\end{document}